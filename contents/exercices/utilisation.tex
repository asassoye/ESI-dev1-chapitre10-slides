\stepcounter{exercice}
\begin{frame}{Exercice \theexercice}{Syllabus Exercice 97}
    Écrire un algorithme qui déclare un tableau de 100 entiers
    et y met les nombres de 1 à 100.
\end{frame}

\begin{frame}{Exercice \theexercice~- Solution}{Syllabus Exercice 97}
    \lstinputlisting{java/AlgorithmeInitialisation.java}
\end{frame}

\stepcounter{exercice}
\begin{frame}{Exercice \theexercice}{Syllabus Exercice 98}
    Expliquez la différence entre $tab[i] = tab[i+1]$
    et $tab[i] = tab[i]+1$.
\end{frame}

\begin{frame}{Exercice \theexercice~- Solution}{Syllabus Exercice 98}
    \begin{block}{\textbf{$tab[i] = tab[i+1]$}}
        Ici, la valeur de $tab[i+1]$ est affecté à $tab[i]$
    \end{block}

    \begin{block}{\textbf{$tab[i] = tab[i]+1$}}
        Ici, la valeur de $tab[i]$ est incrémenté puis est affecté à $tab[i]$
    \end{block}

    \begin{exampleblock}{Equivalence}
        $tab[i] = tab[i]+1$ reviendrait au meme que de faire
        \begin{itemize}
            \item tab[i]++
        \end{itemize}
    \end{exampleblock}
\end{frame}

\stepcounter{exercice}
\begin{frame}[shrink]{Exercice \theexercice}{Syllabus Exercice 99}
    Écrire les entêtes (et uniquement les entêtes)
    des algorithmes qui résolvent les problèmes suivants~:
    \begin{enumerate}
        \item
        Inverser le signe de tous les éléments négatifs dans un tableau d’entiers.
        \item
        Donner le nombre d’éléments négatifs dans un tableau d’entiers.
        \item
        Déterminer si un tableau d’entiers contient au moins un nombre négatif.
        \item
        Déterminer si un tableau de chaines contient
        une chaine donnée en paramètre.
        \item
        Déterminer si un tableau de chaines contient
        au moins deux occurrences de la même chaine,
        quelle qu’elle soit.
        \item
        Retourner un tableau donnant les $n$ premiers nombres premiers,
        où $n$ est un paramètre de l’algorithme.
        \item
        Reçoit un tableau d’entiers
        et retourne un tableau de booléens de la même taille
        où la case $i$ indique si oui ou non
        le nombre reçu dans la case $i$ est strictement positif.
    \end{enumerate}
\end{frame}

\begin{frame}[allowframebreaks]{Exercice \theexercice~- Solution}{Syllabus Exercice 99}
    \begin{enumerate}
        \item
        Inverser le signe de tous les éléments négatifs dans un tableau d’entiers.
        \lstinputlisting[firstline=2, lastline=2, firstnumber=2, morekeywords=String]{java/SyllabusExercice99.java}
        \item
        Donner le nombre d’éléments négatifs dans un tableau d’entiers.
        \lstinputlisting[firstline=3, lastline=3, firstnumber=3, morekeywords=String]{java/SyllabusExercice99.java}
        \item
        Déterminer si un tableau d’entiers contient au moins un nombre négatif.
        \lstinputlisting[firstline=4, lastline=4, firstnumber=4, morekeywords=String]{java/SyllabusExercice99.java}
        \item
        Déterminer si un tableau de chaines contient
        une chaine donnée en paramètre.
        \lstinputlisting[firstline=5, lastline=5, firstnumber=5, morekeywords=String]{java/SyllabusExercice99.java}
        \item
        Déterminer si un tableau de chaines contient
        au moins deux occurrences de la même chaine,
        quelle qu’elle soit.
        \lstinputlisting[firstline=6, lastline=6, firstnumber=6, morekeywords=String]{java/SyllabusExercice99.java}
        \item
        Retourner un tableau donnant les $n$ premiers nombres premiers,
        où $n$ est un paramètre de l’algorithme.
        \lstinputlisting[firstline=7, lastline=7, firstnumber=7, morekeywords=String]{java/SyllabusExercice99.java}
        \item
        Reçoit un tableau d’entiers
        et retourne un tableau de booléens de la même taille
        où la case $i$ indique si oui ou non
        le nombre reçu dans la case $i$ est strictement positif.
        \lstinputlisting[firstline=8, lastline=8, firstnumber=8, morekeywords=String]{java/SyllabusExercice99.java}
    \end{enumerate}

\end{frame}

\stepcounter{exercice}
\begin{frame}{Exercice \theexercice}{Syllabus Exercice 100}
    Écrire un algorithme qui
    inverse le signe de tous les éléments négatifs dans un tableau d’entiers.
\end{frame}

\begin{frame}{Exercice \theexercice~- Solution}{Syllabus Exercice 100}
    On parcours le tableau pour y trouver des nombres négatifs.
    \lstinputlisting[firstline=2, lastline=8, firstnumber=2]{java/SyllabusExercice100.java}
    Si le nombre est négatif alors on l'inverse.
\end{frame}

\stepcounter{exercice}
\begin{frame}{Exercice \theexercice}{Syllabus Exercice 103}
    Écrire un algorithme qui
    donne le nombre d’éléments négatifs dans un tableau d’entiers.
\end{frame}
