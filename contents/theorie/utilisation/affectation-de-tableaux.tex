\begin{frame}{Affectation d'une valeur}
    \pause
    Prenons cet exemple
    \lstinputlisting[firstline=5, lastline=8, firstnumber=5]{java/Affectation.java}

    \pause
    Cela donne
    \lstinputlisting[firstline=1, lastline=1, firstnumber=1, language=bash]{java/Affectation.txt}
\end{frame}

\begin{frame}{Affectation d'un tableau}
    Faisons la même chose avec un tableau et modifions une valeur de la copie du tableau
    \lstinputlisting[firstline=10, lastline=14, firstnumber=10]{java/Affectation.java}

    \pause
    Cela donne
    \lstinputlisting[firstline=2, lastline=3, firstnumber=2, language=bash]{java/Affectation.txt}

\end{frame}

\begin{frame}{Référence d'un tableau}
    Nous avons les mêmes valeurs car quand on crée un tableau et qu'on l'affecte a une variable,
    nous donnons en réalité la référence vers le tableau.

    \pause
    \begin{exampleblock}{Arrays.toString()}
        Nous ne pouvons par conséquent pas afficher immediatement le tableau,
        nous ne ferions qu'afficher la référence du tableau.

        \pause
        La méthode $Arrays.toString()$ permet de créer facilement une chaine de caractère à partir d'un tableau.

        \pause
        \tiny\center Noubliez-pas d'importer la classe adéquate!
        \lstinputlisting[firstline=1, lastline=1, firstnumber=1]{java/Affectation.java}
    \end{exampleblock}
\end{frame}