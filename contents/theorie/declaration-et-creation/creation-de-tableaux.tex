\begin{frame}{Création par l'operateur new}
    \pause
    En Java, les tableaux sont des objets. Pour créer un nouvel objet on utilise l'opérateur $new$ suivi du nom de la classe\\
    \tiny Plus d'informations sur la programmation orienté objet seront données au cours DEV2 \normalsize

    \lstinputlisting[firstline=9, lastline=9, firstnumber=9, morekeywords=Scanner]{java/OperateurNew.java}
    n étant un nombre naturel. Utiliser un nombre négatif déclenchera une exception!

    \pause
    \begin{alertblock}{Attention}
        Une fois la taille d'un tableau d\'efinie, elle ne pourra pas être modifiée!
    \end{alertblock}
\end{frame}

\begin{frame}{Création par l'operateur new}{Code complet}
    \lstinputlisting[morekeywords=Scanner]{java/OperateurNew.java}
\end{frame}

\begin{frame}{Utilisation d'un initialiseur}
    \pause
    En Java, nous pouvons immédiatement initialiser les valeurs d'un tableau.
    Dans ce cas, Java sait automatiquement quelle est la taille du tableau.
    \lstinputlisting[morekeywords=Scanner]{java/Initialiseur.java}

    \pause
    \begin{exampleblock}{Expression comme valeur}
        Nous pouvons utiliser des expressions comme valeurs.
        Celles-ci seront évaluées avant d'être sauvegardées en mémoire.
    \end{exampleblock}

\end{frame}

\begin{frame}{Utilisation d'un initialiseur}{Variante sans initialiseur}
    Voici à quoi ressemblerait le code précédent sans initialiseur.
    \lstinputlisting[morekeywords=Scanner]{java/SansInitialiseur.java}
\end{frame}