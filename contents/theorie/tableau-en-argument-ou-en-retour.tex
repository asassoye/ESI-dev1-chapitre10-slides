\begin{frame}{Tableau en argument}
    \pause
    Reprenons l'exemple précédent. Nous pouvons crée une méthode qui affiche systematiquement tous les noms d'un tableau
    \lstinputlisting[firstline=2, lastline=6, firstnumber=2]{tableau-en-argument-ou-en-retour/java/TableauEnArgument.java}

    \pause
    \begin{alertblock}{Référence au tableau}
        En réalité, on ne transmets pas une copie du tableau à la méthode. On transmet la référence au tableau.
        Concrètement, si on modifie une valeur du tableau dans la méthode,
        le tableau original aura aussi été soumis à la modification.
    \end{alertblock}
\end{frame}

\begin{frame}{Tableau en retour}
    \pause
    Comme on peut référencer un tableau en argument d'une méthode,
    nous pouvons aussi renvoyer une référence vers un tableau en retour d'une méthode.
    \lstinputlisting[firstline=2, lastline=4, firstnumber=4]{tableau-en-argument-ou-en-retour/java/TableauEnRetour.java}

    \pause
    \center\tiny On est d'accord, cette méthode ne sert à rien
\end{frame}

\stepcounter{exercice}
\begin{frame}[shrink]{Exercice \theexercice}{Syllabus Exercice 99}
    \pause
    Écrire les entêtes (et uniquement les entêtes)
    des algorithmes qui résolvent les problèmes suivants~:
    \begin{enumerate}
        \item\pause
        Inverser le signe de tous les éléments négatifs dans un tableau d’entiers.
        \item\pause
        Donner le nombre d’éléments négatifs dans un tableau d’entiers.
        \item\pause
        Déterminer si un tableau d’entiers contient au moins un nombre négatif.
        \item\pause
        Déterminer si un tableau de chaines contient
        une chaine donnée en paramètre.
        \item\pause
        Déterminer si un tableau de chaines contient
        au moins deux occurrences de la même chaine,
        quelle qu’elle soit.
        \item\pause
        Retourner un tableau donnant les $n$ premiers nombres premiers,
        où $n$ est un paramètre de l’algorithme.
        \item\pause
        Reçoit un tableau d’entiers
        et retourne un tableau de booléens de la même taille
        où la case $i$ indique si oui ou non
        le nombre reçu dans la case $i$ est strictement positif.
    \end{enumerate}
\end{frame}

\begin{frame}[allowframebreaks]{Exercice \theexercice~- Solution}{Syllabus Exercice 99}
    \begin{enumerate}
        \item
        Inverser le signe de tous les éléments négatifs dans un tableau d’entiers.
        \lstinputlisting[firstline=2, lastline=2, firstnumber=2, morekeywords=String]{tableau-en-argument-ou-en-retour/java/SyllabusExercice99.java}
        \item
        Donner le nombre d’éléments négatifs dans un tableau d’entiers.
        \lstinputlisting[firstline=3, lastline=3, firstnumber=3, morekeywords=String]{tableau-en-argument-ou-en-retour/java/SyllabusExercice99.java}
        \item
        Déterminer si un tableau d’entiers contient au moins un nombre négatif.
        \lstinputlisting[firstline=4, lastline=4, firstnumber=4, morekeywords=String]{tableau-en-argument-ou-en-retour/java/SyllabusExercice99.java}
        \item
        Déterminer si un tableau de chaines contient
        une chaine donnée en paramètre.
        \lstinputlisting[firstline=5, lastline=5, firstnumber=5, morekeywords=String]{tableau-en-argument-ou-en-retour/java/SyllabusExercice99.java}
        \item
        Déterminer si un tableau de chaines contient
        au moins deux occurrences de la même chaine,
        quelle qu’elle soit.
        \lstinputlisting[firstline=6, lastline=6, firstnumber=6, morekeywords=String]{tableau-en-argument-ou-en-retour/java/SyllabusExercice99.java}
        \item
        Retourner un tableau donnant les $n$ premiers nombres premiers,
        où $n$ est un paramètre de l’algorithme.
        \lstinputlisting[firstline=7, lastline=7, firstnumber=7, morekeywords=String]{tableau-en-argument-ou-en-retour/java/SyllabusExercice99.java}
        \item
        Reçoit un tableau d’entiers
        et retourne un tableau de booléens de la même taille
        où la case $i$ indique si oui ou non
        le nombre reçu dans la case $i$ est strictement positif.
        \lstinputlisting[firstline=8, lastline=8, firstnumber=8, morekeywords=String]{tableau-en-argument-ou-en-retour/java/SyllabusExercice99.java}
    \end{enumerate}

\end{frame}

\stepcounter{exercice}
\begin{frame}{Exercice \theexercice}{Syllabus Exercice 101}
    Écrire un algorithme qui reçoit en paramètre le tableau $integers$
    de $n$ entiers et qui retourne la somme de ses éléments.
\end{frame}

\stepcounter{exercice}
\begin{frame}{Exercice \theexercice}{Syllabus Exercice 104}
    \begin{enumerate}
        \item Écrire un algorithme qui reçoit en paramètre le tableau
        $cotes$ de $n$ entiers représentant les cotes des étudiants
        et qui retourne un booléen indiquant s’il contient \textbf{au
        moins} une fois la valeur 20.

        \item\pause Écrire un algorithme qui reçoit en paramètre le tableau
        $cotes$ de $n$ entiers représentant les cotes des étudiants
        et qui retourne un booléen indiquant s’il contient
        \textbf{exactement} une fois la valeur 20.

    \end{enumerate}
\end{frame}